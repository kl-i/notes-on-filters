\documentclass[main.tex]{subfiles}

\begin{document}

\begin{dfn}[Topological Space]
  
  Let $X$ be a set.
  A \emph{topology on $X$} consists of the following data : 
  \begin{itemize}
    \item (Set of Opens) a subset $\tau_X \subs \SUB\SET(X)$.
    Subsets $U$ of $X$ that are in $\tau_X$ are called \emph{opens}.
    \item (Finite Intersection) 
    For $I \subs \tau_X$ finite, $\bigcap_{U \in I} U \in \tau_X$.
    In particular, $X = \bigcap_{U \in \nothing} U \in \tau_X$.
    \item (Arbitrary Union)
    For $I \subs \tau_X$, $\bigcup_{U \in I} U \in \tau_X$.
    In particular, $\nothing = \bigcup_{U \in \nothing} U \in \tau_X$.
  \end{itemize}
  A \emph{topological space} is 
  a set $X$ together with a topology on it.
  We often write $X$ instead of $(X,\tau_X)$ for a topological space. 

  For a sequence $a : \N \to X$ and $x \in X$,
  we write $a_n \to x$ and say \emph{$a$ converges to $x$} when 
  for all $x \in U \subs X$ where $U$ is open, 
  there exists $N \in \N$ such that $a \N_{\geq N} \subs U$.
\end{dfn}

\begin{rmk}[On the standard definition of a topological space.]
  
  The above definition is standard and can be motivated as
  the abstraction of opens in $\R^n$. 
  However, this begs the question of why consider opens in $\R^n$. 
  The idea of ``getting close to a point'' is arguably 
  more intuitively captured by the notion of sequences converging. 
  The view on \emph{filters} that I adopt is that 
  they are the generalisation of sequences for topological spaces. 
  \begin{align*}
    \text{$\R^n$} &\to \text{Topological Spaces} \\
    \text{Sequences} &\to \text{Filters}
  \end{align*}
\end{rmk}

\begin{prop}[Filter of a Sequence]
  
  Let $X$ be a set, $a : \N \to X$. 
  For a subset $V \subs X$,
  we say \emph{$a$ converges to $V$} when
  there exists $N \in \N$ such that 
  for all $n \geq N$, $a_n \in V$.
  Let $\al$ be the set of subsets of $X$ that $a$ converges into. 
  Then we have the following : 
  \begin{itemize}
    \item (Universe) $X \in \al$.
    \item (Finite Intersection) For $U, V \in \al$, $U \cap V \in \al$.
    \item (Upwards Closed) For $U \subs V \subs X$, 
    $U \in \al$ implies $V \in \al$.
  \end{itemize}
  $\al$ is called the \emph{eventuality filter of $a$}.
  We will simply call it the \emph{filter of $a$}.
\end{prop}
\begin{proof}
  Easy.
\end{proof}

\begin{rmk}
  For purposes of convergence, 
  the only data of a sequence we really care about is 
  which subsets they converge into. 
  The filter of a sequence extracts this data from a sequence and 
  one can think of a filter in general as abstracting this.
\end{rmk}

\begin{dfn}[Filter] 

  Let $X$ be a set and $\al \subseteq 2^X$. 
  Then $\al$ is a \emph{filter on $X$} when 
  \begin{itemize}
    \item (Universe) $X \in \al$.
    \item (Finite Intersection) For $U, V \in \al$, $U \cap V \in \al$.
    \item (Upwards Closed) For $U \subs V \subs X$, 
    $U \in \al$ implies $V \in \al$.
  \end{itemize}

  For a filter $\al$ on $X$ and $V \subs X$,
  we write $\al \to V$ and say \emph{$\al$ converges into $V$} 
  when $V \in \al$.

  We will use $\fil(X)$ denote the set of all filters on $X$.
  Define $\be \to \al$ to mean $\be \sups \al$,
  yielding a partial order $\to$ on $\fil(X)$.

  The minimal and maximal filters with respect to $\to$ are respectively 
  the powerset of $X$ and the filter $\set{X}$.
  We will denote these $\bot, \top$ and call them 
  the \emph{initial and terminal filter} respectively.
  
\end{dfn} 

\begin{rmk}[On the initial and terminal filter]
  
  Let $X$ be a set and $\al \in \fil(X)$.
  Then $\al = \bot$ if and only if $\al \to \nothing$.
  You can thus think of $\bot$ as the ``empty sequence''.

  On the other extreme, for any $A \subs X$,
  $\top \to A$ if and only if $A = X$.
  You can think of $\top$ as the ``chaotic sequence'',
  which ``visits every subset of $X$ but never converges into any''.
\end{rmk}

\begin{rmk}[On the Direction of Partial Order of Filters]
  If $b : \N \to X$ be a subsequence of $a$,
  then $\be \to \al$ where $\al,\be$ are filters of $a,b$ respectively.
  So for general filters $\al,\be$,
  you can think of $\be \to \al$ as saying 
  ``$\be$ is a subsequence of $\al$''.
  The next proposition says that
  for a point $x \in X$ where $X$ is a topological space, 
  there's is a ``largest sequence converging to $x$''.
\end{rmk}

\begin{prop}[Neighbourhood Filter]
  
  Let $X$ be a topological space, $x \in X$. 
  For a subset $V \subs X$,
  we say $V$ is a \emph{neighbourhood of $x$} when 
  there exists an open $U$ of $X$ such that 
  $x \in U \subs V$.
  The \emph{neighbourhood filter of $x$}, denoted $N(x)$,
  is then defined to be the set of neighbourhoods of $x$.
  We then have the following : 
  \begin{itemize}
    \item $N(x)$ is a filter on $X$.
    \item For all sequences $a : \N \to X$,
    $a_n \to x$ if and only if $\al \sups N(x)$ where
    $\al$ is the filter of $a$.
  \end{itemize}
  Hence, for a filter $\al$ on $X$,
  we write $\al \to x$ and say \emph{$\al$ converges to x} when 
  $\al \sups N(x)$.
\end{prop}
\begin{proof} 
  Easy. 
\end{proof}

\begin{rmk}
  One may wonder if it is possible to 
  give a topological space by choosing a neighbourhood filter at each point.
  The answer is yes. 
\end{rmk}

\begin{prop}[Topological Spaces by Neighbourhood Filters]
  
  Let $X$ be a set. 
  Define a \emph{system of neighbourhood filters on $X$} by the following data :
  \begin{itemize}
    \item (The Neighbourhood Filters) 
    A function $N : X \to \fil(X)$
    \item (Centred) 
    For each $x \in X$ and $U \in N(x)$, $x \in U$.
    \item (Mutual Neighbourhoods)
    For each $x \in X$ and $V \in N(x)$,
    there is a $U \subs V$ such that 
    $x \in U$ and for all $y \in U$, $U \in N(y)$.
  \end{itemize}

  Let $N$ be a system of neighbourhood filters on $X$ and 
  $\tau$ a topology in $X$.
  Consider the following constructions : 
  \begin{itemize}
    \item Define the \emph{topology associated to $N$} by 
    declaring $U \subs X$ to be open when 
    for all $x \in U$, $U \in N(x)$.
    Then this forms a topology on $X$.
    \item Define the \emph{system of neighbourhood filters on $X$
    associated to $\tau$} by
    assigning to each point $x$ its neighbourhood filter as defined before.
    This is a system of neighbourhood filters on $X$.
  \end{itemize}
  Then the above two processes are inverses,
  yielding a bijection between systems of neighbourhood filters and 
  topologies on $X$.
  
\end{prop}
\begin{proof}
  Easy. 
\end{proof}

\end{document}