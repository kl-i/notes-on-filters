\documentclass[main.tex]{subfiles}

\begin{document}

\begin{rmk}[Naive Characterisation of Compactness]
  
  Let $X$ be a topological space. 
  Then the following are equivalent : 
  \begin{enumerate}
    \item $X$ compact. 
    \item For all sequences $\al$ in $X$, 
    there exists $x \in X$ such that $x$ is a limit point of $\al$.
    \item For all sequences $\al$ in $X$, 
    there exists a sequence $\be$ and $x \in X$,
    such that $\be$ is a subsequence of $\al$ and $\be\to x$.
  \end{enumerate}
  Note that given a chain of subsequences 
  $\cdots \to \al_1 \to \al_0 \to \al$, 
  it is not possible to have ``$\al_\infty$''.
  We will see that with filters, 
  this is possible,
  i.e. we can have ``sequences such that all subsequences are either 
  empty or itself''.
  Calling these ``\emph{minimal sequences}'', 
  we should have another equivalence : 
  \begin{enumerate}[resume]
    \item For all ``minimal sequences'' $\al$ in $X$,
    there exists a point $x \in X$ such that $\al \to x$.
  \end{enumerate}
\end{rmk}

\begin{dfn} [Limit Point of a Filter]
  
  Let $X$ be a topological space, $\al \in \fil(X)$, $x \in X$.
  Then $x$ is a \emph{limit point of $\al$} when
  for all $V \in F$, $x$ is a limit point of $V$.
\end{dfn}

\begin{prop} [Characterisation of Limit Points of a Filter]
  
  Let $X$ be a topological space, $\al \in \fil(X)$, $x \in X$. 
  Then TFAE : 
  \begin{enumerate}
    \item $x$ is a limit point of $F$.
    \item (Converging Subfilter) There exists $\be \in \fil(X)$ such that 
    $\bot \neq \be \to \al$ and $\be \to x$.
    \item (Neighbourhood Filter Suffices) $\sqcap\set{N(x),F} \neq \bot$.
  \end{enumerate}
\end{prop}
\begin{proof}
  $(1\iff 3)$
  \begin{align*}
    N(x) \sqcap \al \neq \bot
    &\iff \varnothing \notin N(x) \sqcap \al
    \iff \forall\, U \in N(x), \forall\, V \in \al, U \cap V \neq \varnothing. \\
    &\iff \forall\, V \in \al, x \in \bar{V}. 
    \iff x \text{ limit point of } \al
  \end{align*} 

  $(2\iff 3)$ Follows from definitions of convergence and join of filters.
\end{proof}

\begin{dfn} [Minimal Filters]
  
  Let $X$ be a set and $\al \in \fil(X)$.
  Then $\al$ is a \emph{minimal} when 
  $\bot \neq \al$ and for all $\be \in \fil(X)$, 
  $\be \to \al$ implies $\bot = \be$ or $\be = \al$.
  \footnote{
    In the literature,
    it is common to not use the dual partial order $\to$ as we have,
    resulting in these filters being \emph{maximal} instead of minimal.
    Hence they are more commonly called 
    \emph{ultrafilters}.
  }
\end{dfn}

\begin{prop} [Characterisation of Minimal filters]
  
  Let $X$ be a set and $\al \in \fil(X)$.
  Then TFAE : 
  \begin{enumerate}
    \item $\al$ is minimal.
    \item (``$\al$ is decisive'') 
    For all $A \subseteq X$, $\al \to A$ or $\al \to X\minus A$.
  \end{enumerate}
\end{prop}
\begin{proof}
  $(1\imp 2)$ 
  Let $A \subseteq X$, $\lift{A}{} : A \to X$ the inclusion.
  % Consider $\sqcup\set{\io_A\set{A},F}$.
  % Note that $F \subseteq \sqcup\set{\io_A\set{A},F}$,
  % so by maximality of $F$, 
  % $F = \sqcup\set{\io_A\set{A},F}$ or $\sqcup\set{\io_A\set{A},F} = 2^X$.
  % In the first case, 
  % by definition of join and $\io_A\set{A}$, 
  % $A \in F$.
  % In the second case, 
  % $\varnothing \in \sqcup\set{\io_A\set{A},F}$,
  % so there exists $U \supseteq A$ and $V \in F$ such that 
  % $U \cap V = \varnothing$.
  % In particular, $V \subseteq X\minus A$, 
  % which implies $X\minus A \in F$.
  Then $\lift{A}{}\set{A} \sqcap \al \to \al$ so 
  either $\bot = \lift{A}{}\set{A} \sqcap \al$ or 
  $\lift{A}{}\set{A} \sqcap \al = \al$.
  In the first case, there exists $V \subs X$, $\al \to V$ 
  with $A \cap V = \nothing$. 
  Hence $\al \to X\minus A$.
  In the second case, $\al \to \lift{A}{}\set{A}$ gives $\al \to A$.

  $(2\imp 1)$ 
  Let $\be \in \fil(X)$ with $\be \to \al$.
  Suppose $\be \neq \al$, i.e. there exists $V \subs X$ such that 
  $\be \to V$ and $\al \not\to V$.
  Then $\al \to X\minus V$ by assumption,
  whence $\be \to V$ and $\be \to X\minus V$, i.e. $\bot = \be$.
\end{proof}

\begin{prop} [Characterisation of Compactness]
  
  Let $X$ be a topological space. 
  Then TFAE : 
  \begin{enumerate}
    \item (Standard Definition) for all $\UU \subs \tau_X$,
    $X \subs \bigcup_{U \in \UU} U$ implies 
    there exists finite $\UU_0 \subs \UU$ such that 
    $X \subs \bigcup_{U \in \UU_0} U$.
    \item (Closed Variant of Standard Definition)
    for all sets $I$ of closed subsets of $X$,
    if for all finite $I_0 \subs I$, $\bigcap_{V \in I_0} V \neq \nothing$
    then $\bigcap_{V \in I} V \neq \nothing$.
    \item (Limit Point of Filters) For all $\al \in \fil(X)$,
    $\bot \neq \al$ implies 
    there exists $x \in X$ such that $x$ is a limit point of $\al$.
    \item (Convergent Subfilter) For all $\al \in \fil(X)$,
    $\bot \neq \al$ implies 
    there exists $\be \in \fil(X)$ and $x \in X$
    such that $\bot \neq \be \to \al$ and $\be \to x$.
    \item (Minimal Filters Convergent) For all minimal $\al \in \fil(X)$,
    there exists $x \in X$, $\al \to x$.
  \end{enumerate}
  $X$ is called \emph{compact} when any (and thus all)
  of the above are true. 
\end{prop}
\begin{proof}

  $(1 \iff 2)$ Easy. $(2 \imp 3)$ 
  Let $\bot \neq \al \in \fil(X)$.
  Define $\bar{\al} := \set{\bar{V}\st \al \to V}$.
  Then $\bot \neq \al$ implies 
  all finite intersections of sets in $\bar{\al}$ are non-empty. 
  Hence $\bigcap_{\bar{V} \in \bar{\al}} \bar{V} \neq \nothing$,
  giving a limit point of $\al$.

  $(3\imp 2)$ Let $I$ be a set of closed subsets of $X$
  such that for all finite $J \subseteq I$, 
  $\bigcap_{V \in J} V \neq \varnothing$.
  Take the ``filter generated by $I$''.
  Since every $V \in I$ is closed, 
  limit points will be precisely elements of $V$ in $I$.
  % Let \[
  %   F_I := \set{
  %     W \subseteq C \st \exists\, J\text{ fin} \subseteq I, 
  %     \bigcap_{V \in J} V \subseteq W
  %   }
  % \]
  % Then $F_I \neq 2^X \in \fil(X)$.
  % So there exists $x \in X$
  % such that $x$ is a limit point of $F_I$.
  % Then for all $V \in I \subseteq F_I$, 
  % $V$ closed $\imp x \in V$,
  % i.e. $\cap_{V \in I} V \neq \varnothing$.

  $(3 \iff 4)$ By characterisation of limit points.

  $(4 \iff 5)$ 
  Forward is clear. For backwards, we use choice :  
  \begin{lem} [Existence of Maximal Filters]
  
    Let $X$ be a set, $\bot \neq \al_0 \in \fil(X)$.
    Then there exists $\al \in \fil(X)$ such that 
    $\al \to \al_0$ and $\al$ is minimal.
    \begin{proof1}
      Standard application of Zorn's lemma.
    \end{proof1}
  \end{lem}
\end{proof}

\begin{rmk}
  The following is a (now) almost trivial proof 
  of Tychonoff's theorem.
\end{rmk}

\begin{prop} [Product of Compact is Compact (Tychonoff)]
  
  Let $I$ be a set and for $i \in I$, $X_i$ a topological space.
  Then $\prod_{i \in I} X_i$ is compact $\iff$
  for all $i \in I$, $X_i$ is compact. 
\end{prop}
\begin{proof}
  $(\imp)$ 
  \begin{lem}[Image of compact is compact under continuous maps]

    Let $\pi \in \TOP(X,Y)$.
    Then $X$ compact implies $\pi X$ compact (with subspace topology).
    \begin{proof1}
      It follows from the induced topology that 
      $\pi : X \to \pi X$ is continuous.
      Let $\al \in \fil(\pi X)$ be minimal. 
      Since $\pi : X \to \pi X$ is surjective,
      $\bot \neq \al$ implies $\bot \neq \ph\inv \al$.
      By compactness of $X$, there exists $\be \in \fil(X)$ such that 
      $\bot \neq \be \to \pi\inv \al$ and a point $x \in X$ with $\be \to x$.
      Then $\bot \neq \pi\be \to \al$ and $\pi \be \to \pi(x)$,
      where $\al = \pi\be$ by minimality, giving a point in $\pi X$
      that $\al$ converges to as desired. 
    \end{proof1}
  \end{lem}

  $(\limp)$ 
  It suffices that all minimal filters on $\prod_{i \in I} X_i$ converge.
  So let $\al \in \fil\brkt{\prod_{i \in I} X_i}$ be minimal.
  We hope that for all $i \in I$, $\pi_i F \in \fil(X_i)$ is minimal so 
  we can use compactness of the components. 
  Indeed : 
  \begin{lem} [Image of Minimal Filters]
  
    Let $X,Y$ be sets, $f : X \to Y$, $\al \in \fil(X)$ minimal.
    Then $f \al$ is a minimal filter of $Y$. 
  \end{lem}
  \begin{proof1}
    It suffices for all $B \subs Y$, $f \al \to B$ or $f \al \to Y\minus B$.
    This is clear since any $B \subs Y$ partitions $X$ by 
    $f\inv B$ and $f\inv (Y\minus B)$.
  \end{proof1}
  Then exists $x_i \in X_i$ such that $\fall{}{X_i} \al$ converges to $x_i$
  where $\fall{}{X_i}$ is projection into the $i$-th component.
  By the axiom of choice, $x = (x_i)_{i\in I} \in \prod_{i \in I} X_i$.
  Then by the induced topology, $\al \to x$.
\end{proof}

\end{document}