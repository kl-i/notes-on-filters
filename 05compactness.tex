\documentclass[main.tex]{subfiles}

\begin{document}
\section{Compactness}

\begin{rmk} Naive Characterisation of Compactness. 
  
  Let $X$ be a topological space. 
  Then the following are equivalent : 
  \begin{enumerate}
    \item $X$ compact. 
    \item For all sequences $\al$ in $X$, 
    there exists $x \in X$ such that $x$ is a limit point of $\al$.
    \item For all sequences $\al$ in $X$, 
    there exists a sequence $\be$ and $x \in X$,
    such that $\be$ is a subsequence of $\al$ and $\be$ converges to $x$.
  \end{enumerate}
\end{rmk}

\begin{dfn} Limit Point of a Filter.
  
  Let $X$ be a topological space, $F \in Fil(X)$, $x \in X$.
  Then $x$ is a \emph{limit point of $F$} when
  for all $V \in F$, $x$ is a limit point of $V$.
\end{dfn}

\begin{thm} Characterisation of Limit Points of a Filter.
  
  Let $X$ be a topological space, $F \in Fil(X)$, $x \in X$. 
  Then the following are equivalent : 
  \begin{enumerate}
    \item $x$ is a limit point of $F$.
    \item $\sqcup\set{N(x),F} \neq 2^X$.
    \item There exists $G \in Fil(X)$ such that 
    $F \subseteq G \neq 2^X$ and $G$ converges to $x$.
  \end{enumerate}
\end{thm}
\begin{proof}
  $(1\iff 2)$
  \begin{align*}
    \sqcup\set{N(x),F} \neq 2^X 
    &\iff \varnothing \notin \sqcup\set{N(x),F}
    \iff \forall\, U \in N(x), \forall\, V \in F, U \cap V \neq \varnothing. \\
    &\iff \forall\, V \in F, x \in \bar{V}. 
    \iff \forall\, V \in F, x \text{ limit point of } V.
    \iff x \text{ limit point of } F
  \end{align*} 

  $(2\iff 3)$ Follows from definitions of convergence and join of filters.
\end{proof}

\begin{thm} Characterisation of Compactness. 
  
  Let $X$ be a topological space. 
  Then the following are equivalent : 
  \begin{enumerate}
    \item $X$ compact. 
    \item For all $F \in Fil(X)$,
    $F \neq 2^X \imp$
    there exists $x \in X$ such that $x$ is a limit point of $F$.
    \item For all $F \in Fil(X)$,
    $F \neq 2^X \imp$
    there exists $G \in Fil(X)$ and $x \in X$
    such that $F \subseteq G \neq 2^X$ and $G$ converges to $x$.
  \end{enumerate}
\end{thm}
\begin{proof}
  $(2 \iff 3)$ Follows from characterisation of limit points of a filter.

  $(1\imp 2)$ 
  Let $F \neq 2^X \in Fil(X)$.
  Define $\bar{F} := \set{\bar{V}\st V \in F}$.
  Then $F \neq 2^X$ implies 
  all finite intersections of sets in $\bar{F}$ are non-empty. 
  Since $X$ is compact, 
  we then have $\bigcap_{\bar{V} \in \bar{F}} \bar{V} \neq \varnothing$.
  Let $x \in \bigcap_{\bar{V} \in \bar{F}} \bar{V}$.
  Then for all $V \in F$, $x$ is a limit point of $V$.

  $(2\imp 1)$ Let $I$ be a set of closed subsets of $X$
  such that for all finite $J \subseteq I$, 
  $\bigcap_{V \in J} V \neq \varnothing$.
  Let \[
    F_I := \set{
      W \subseteq C \st \exists\, J\text{ fin} \subseteq I, 
      \bigcap_{V \in J} V \subseteq W
    }
  \]
  Then $F_I \neq 2^X \in Fil(X)$.
  So there exists $x \in X$
  such that $x$ is a limit point of $F_I$.
  Then for all $V \in I \subseteq F_I$, 
  $V$ closed $\imp x \in V$,
  i.e. $\cap_{V \in I} V \neq \varnothing$.
\end{proof}

\begin{dfn} Maximal filters. 
  
  Let $X$ be a set and $F \in Fil(X)$.
  Then $F$ is a \emph{maximal} when 
  $F \neq 2^X$ and for all $G \in Fil(X)$, 
  $F \subseteq G \imp F = G$ or $G = 2^X$. 
\end{dfn}

\begin{lem} Existence of Maximal Filters.
  
  Let $X$ be a set, $F \in Fil(X)$ and $F \neq 2^X$.
  Then there exists $G \in Fil(X)$ such that 
  $F \subs G$ and $G$ is maximal.
\end{lem}
\begin{proof}
  Standard application of Zorn's lemma.
\end{proof}

\begin{cor} Maximal Filters Characterisation of Compactness.
  
  Let $X$ be a topological space. 
  Then $X$ is compact $\iff$ for all $F \in Fil(X)$, 
  $F$ maximal implies the existence of $x \in X$ such that 
  $F$ converges to $x$.
\end{cor}
\begin{proof}
  $(\imp)$ Let $F \in Fil(X)$ be maximal. 
  Then by the characterisation of compactness, 
  there exists $G \in Fil(X)$ and $x \in X$ such that
  $F \subs G \neq 2^X$ and $G$ converges to $x$.
  By maximality of $F$, $F = G$.

  $(\limp)$ By the characterisation of compactness,
  it suffices to show that for all $F \in Fil(X)$,
  $F \neq 2^X$ implies the existence of 
  $G \in Fil(X)$ and $x \in X$
  such that $F \subs G \neq 2^X$ and $G$ converges to $x$.

  So let $F \in Fil(X)$ and $F \neq 2^X$.
  Then there exists $G \in Fil(X)$ such that 
  $F \subs G$ and $G$ is maximal. 
  There exists $x \in X$ such that $G$ converges to $x$
  and $G \neq 2^X$ by definition.
\end{proof}

\begin{thm} Characterisation of Maximal filters.
  
  Let $X$ be a set and $F \in Fil(X)$.
  Then the following are equivalent : 
  \begin{enumerate}
    \item $F$ is maximal. 
    \item For all $A \subseteq X$, $A \in F$ or $X\setminus A \in F$.
  \end{enumerate}
\end{thm}
\begin{proof}
  $(1\imp 2)$ Let $A \subseteq X$, $\io_A : A \to X$.
  Consider $\sqcup\set{\io_A\set{A},F}$.
  Note that $F \subseteq \sqcup\set{\io_A\set{A},F}$,
  so by maximality of $F$, 
  $F = \sqcup\set{\io_A\set{A},F}$ or $\sqcup\set{\io_A\set{A},F} = 2^X$.
  In the first case, 
  by definition of join and $\io_A\set{A}$, 
  $A \in F$.
  In the second case, 
  $\varnothing \in \sqcup\set{\io_A\set{A},F}$,
  so there exists $U \supseteq A$ and $V \in F$ such that 
  $U \cap V = \varnothing$.
  In particular, $V \subseteq X\minus A$, 
  which implies $X\minus A \in F$.

  $(2\imp 1)$ Let $G \in Fil(X)$ and $F \subseteq G$.
  Suppose there exists $V \in G\minus F$.
  Then $X\minus V \in F \subseteq G$,
  which implies $\varnothing = V \cap \brkt{X\minus V} \in G$,
  i.e. $G = 2^X$.
\end{proof}

\begin{lem} Image of Maximal filters.
  
  Let $X,Y$ be sets, $f : X \to Y$, $F \in Fil(X)$, $F$ maximal filter.
  Then $f F$ is an maximal filter of $Y$. 
\end{lem}
\begin{proof}
  By the characterisation of maximal filters, 
  it suffices for all $B \subs Y$, $B \in fF$ or $Y\minus B \in fF$.
  Let $B \subs Y$.
  Then $f\inv B \in F$ or $X\minus f\inv B \in F$.
  For the first case, $f f\inv B \subs B$ with $f f\inv B \in fF$,
  so $B \in fF$.
  In the latter case, $f(X\minus f\inv B) \subs Y\minus B$ with 
  $f(X\minus f\inv B) \in fF$, so $Y\minus B \in fF$.
\end{proof}

\begin{thm} Product of Compact is Compact (Tychonoff).
  
  Let $I$ be a set and for $i \in I$, $X_i$ a topological space.
  Then $\prod_{i \in I} X_i$ is compact $\iff$
  for all $i \in I$, $X_i$ is compact. 
\end{thm}
\begin{proof}
  $(\imp)$ Image of compact is compact under continuous maps.

  $(\limp)$ By the maximal filters characterisation of compactness,
  it suffices that all maximal filters on $\prod_{i \in I} X_i$ converge.
  So let $F \in Fil\brkt{\prod_{i \in I} X_i}$ be maximal.
  Then for all $i \in I$, $\pi_i F \in Fil(X_i)$ is maximal.
  Since each $X_i$ is compact, 
  there exists $x_i \in X_i$ such that $\pi_i F$ converges to $x_i$.
  By the axiom of choice, 
  $x = (x_i)_{i\in I} \in \prod_{i \in I} X_i$.
  By characterisation of filters on products, 
  $F$ converges to $x$.
\end{proof}

\end{document}