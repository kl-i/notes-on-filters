\documentclass[main.tex]{subfiles}

\begin{document}

\begin{rmk}[Motivation]
  
  Now that we have a category $\TOP$, 
  we hope to be able to construct new topological spaces from 
  existing ones. 
  For example, given $X, Y \in \TOP$,
  we wish to give $X\times Y$ a suitable topology. 
  Let $\fall{}{X} : X\times Y \to X$, $\fall{}{X} : X \times Y \to Y$
  denote the two projections. 
  Then we hope that for any filter $\al$ on $X\times Y$ and $p \in X\times Y$,
  \[
    \al \to p \iff \brkt{\fall{}{X}\al \to \fall{}{X}p \text{ and } 
    \fall{}{Y} \al \to \fall{}{Y} p}
  \]
  But this is if and only if \[
    \al \to {\fall{-1}{X}} N(\fall{}{X}p) \text{ and }
    {\fall{-1}{Y}} N(\fall{}{Y} p)
  \]
  If we could ``intersect'' the two filters 
  ${\fall{-1}{X}} N(\fall{}{X}p),
  {\fall{-1}{Y}} N(\fall{}{Y} p)$ then we can use that as
  the definition of $N(p)$ in $X\times Y$.
  This leads us to the following. 
\end{rmk}

\begin{dfn}[Meet of an arbitrary collection of Filters]
  
  Let $X$ be a set, $F \subs \fil(X)$.
  Define the \emph{meet over $F$} as 
  \[
    \sqcap\, F := \set{
      W \subs X \st \exists\, I\text{ finite } \subs \bigcup_{\al \in F}\al,
      \bigcap_{V \in I} V \subs W
    }
  \]
  This filter has the follow \emph{universal property} : 
  for all $\be \in \fil(X)$,
  $\be \to \sqcap\, F$ if and only if 
  for all $\al \in F$, $\be \to \al$.
  In particular, $\sqcap \nothing = \top$.
\end{dfn}

\begin{prop}[Induced Topology]
  
  Let $(\fall{}{X_i} : X \to X_i)_{i \in I}$ be a set of set-theoretic maps, 
  where each $X_i \in \TOP$.
  For $x \in X$, 
  define \[
    N(x) := \sqcap\,
    \set{{\fall{-1}{X_i}} N(\fall{}{X_i}(x)) \st i \in I}
  \]
  Then this defines a system of neighbourhood filters on $X$,
  making it a topological space with the following property : 
  for all $\al \in \fil\brkt{X}$, 
  $\al \to x$ $\iff$ 
  for all $i \in I$, $\fall{}{X_i} \al \to \fall{}{X_i} x$.

  In particular, 
  for all set-theoretic $f : Y \to X$ where $Y \in \TOP$,
  $f \in \TOP(Y,X)$ if and only if 
  for all $i \in I$, $\fall{}{X_i} \circ f \in \TOP(Y,X_i)$.
\end{prop}
\begin{proof}
  % It suffices to show that $N(x)$ is the minimal filter containing
  % all $\pi_i\inv N(\pi_i(x))$.
% 
  % First note that for all $i \in I$, continuity of $\pi_i$ implies 
  % $N(\pi_i(x)) \subs \pi_iN(x)$,
  % which implies by the Galois connection of filters, 
  % $\pi_i\inv N(\pi_i(x)) \subs N(x)$.
% 
  % Next, let $F \in Fil\brkt{\prod_{i\in I}X_i}$ such that 
  % all $\pi_i\inv N(\pi_i(x)) \subs F$.
  % Let $W \in N(x)$.
  % By definition, 
  % there exists $U \in Open(\prod_{i\in I}X_i)$, $x \in U \subs W$.
  % Then by definition of the product topology, 
  % there exists finite $J \subs I$ and $(U_i) \in \prod_{i \in J} Open(X_i)$
  % such that $x \in \bigcap_{i \in J} \pi_i\inv U_i \subs U$.
  % This implies for all $i\in J$, $\pi_i\inv U_i \in \pi_i\inv N(\pi_i(x))$.
  % Since all $\pi_i\inv U_i \in N(x) \cap F$
  % and $\bigcap_{i\in J}\pi_i\inv U_i \subs W$,
  % we have $W \in F$.
  % This proves $N(x) = \sum_{i \in I} \pi_i\inv N(\pi_i(x))$.
  % 
  % Let $F$ be a filter on the product.
  % Then $F$ converges to $x$ $\iff N(x) \subs F \iff 
  % \forall\, i\in I, \pi_i\inv N(\pi_i(x)) \subs F. \iff 
  % \forall\, i\in I, N(\pi_i(x)) \subs \pi_i F \iff
  % \forall\, i \in I, F$ converges to $\pi_i(x)$ 
  % by the Galois connection of filters. 
  
  $N$ is clearly centered and $N(x)$ is indeed a filter.
  To show existence of mutual neighbourhoods, let $W \in N(x)$.
  By definition, there exists 
  $\set{V_i}_{i\in I_0}$ where $I_0 \subs I$ is finite
  and $V_i \in N(\fall{}{X_i} x)$ with 
  $W \sups \bigcap_{i \in I_0} \fall{-1}{X_i} V_i$.
  By definition of neighbourhood filters, 
  we have $x \in U_i \subs V_i$ for some open $U_i$ for every $i \in I_0$.
  It then follows that $x \in \bigcap_{i \in I_0} U_i \subs W$ and 
  for all $y \in \bigcap_{i \in I_0} U_i$, $\bigcap_{i \in I_0} U_i \in N(y)$.
  The rest follows easily by design. 
\end{proof}

\begin{rmk}
  The above applies in particular to the 
  arbitrary product of topological spaces,
  and making subsets into topological spaces.  
  In these cases, the induced topology is respectively called 
  the \emph{product topology} and the \emph{subspace} topology. 

  I hoped to dualize the above discussion by 
  defining the \emph{join} of an arbitrary collection of filters and 
  use it to define the \emph{co-induced topology} on $X$ from 
  a collection of maps $(X_i \to X)_{i\in I}$ where $X_i \in \TOP$.
  In particular, it should be the case
  for a surjection $\pi : X_0 \to X$ where $X_0 \in \TOP$,
  the co-induced topology is recovers the usual quotient topology.
  However, I was stuck on the following example : 
  \[
    \set{0,1} \overset{\subs}{\to} \set{0,1,2}
  \]
  where all subsets of $\set{0,1}$ are open. 
  I couldn't see an obvious way of defining the neighbourhood filter of $2$,
  and could not find any sources on this.
\end{rmk}
\end{document}