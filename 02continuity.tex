\documentclass[main.tex]{subfiles}

\begin{document}
\section{Continuity}

\begin{rmk}[Naive Characterisation of Continuity]
  
  Let $X, Y$ be topological spaces, $f : X \to Y$ a map of sets. 
  Then the following are equivalent : 
  \begin{enumerate}
    \item $f$ continuous. 
    \item For all $x \in X$ and $\al : \N \to X$,
    $\al$ converges to $x$ $\imp$
    $f \circ \al$ converges to $f(x)$.
  \end{enumerate}
\end{rmk}

\begin{prop} [Neighbourhood Filter]
  
  Let $X$ be a topological space, $x \in X$, 
  $Open(X)$ the set of opens of $X$. \newline
  Define the set of \emph{neighbourhoods of $x$} as 
  \[N(x) := 
  \set{V \subseteq X \st \exists\,U \in Open(X), x \in U \subseteq V}\]
  Then $N(x) \in Fil(X)$. 
\end{prop}
\begin{proof} 
  Clear.
\end{proof}

\begin{dfn} [Filter Converging to Point]
  
  Let $X$ be a topological space, $x \in X$, $F \in Fil(X)$.
  Then \emph{$F$ converges to $x$} when $N(x) \subseteq F$.
\end{dfn}

\begin{rmk} Motivation for Filter Convergence Definition. 
  
  Let $X$ be a topological space, $x \in X$,
  $\al : \N \to X$. 
  Define the \emph{filter associated with $\al$} as 
  \[F_\al := 
  \{V \subseteq X \st \exists\, n \in \N, \al\N_{\geq n} \subseteq V\}\]
  Then $\al$ converges to $x$ if and only if $F_\al$ converges to $x$.
\end{rmk}

\begin{thm} [Characterisation of Continuity]
  
  Let $X, Y$ be topological spaces, $f : X \to Y$ a map of sets. 
  Then the following are equivalent : 
  \begin{enumerate}
    \item $f$ continuous. 
    \item For all $x \in X$ and $F \in Fil(X)$,
    $F$ converges to $x$ $\imp$ $fF$ converges to $f(x)$.
    \item For all $x \in X$, $fN(x)$ converges to $f(x)$.
  \end{enumerate}
\end{thm}
\begin{proof}
  $(1\imp 2)$
  Let $x \in X$, $F \in Fil(x)$, $F$ converges to $x$. 
  Let $V \in N(f(x))$.
  Then $f^{-1}V \in N(x) \subseteq F$ by continuity of $f$,
  and $f f^{-1} V \subseteq V$.
  So $N(f(x)) \subseteq f F$.

  $(2\imp 3)$ Clear. 

  $(2\imp 1)$ 
  Let $U \in Open(Y)$.
  Then for all $x \in f^{-1}U$,
  $fN(x)$ converges to $f(x)$ implies $U \in fN(x)$.
  So there exists $U_x \in N(x)$, $fU_x \subseteq U$.
  It follows that $f^{-1}U = \bigcup_{x \in f^{-1}U} U_x$, 
  which is open. 
\end{proof}

\end{document}